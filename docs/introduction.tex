\chapter{Introduction} \label{cha:introduction}

The purpose of this Collection is to give a definitive listing of Plain Doubles
methods.
It replaces the 1980 Plain Doubles Methods and Variations.

The Central Council's Decision (E) defines four types of method.
This volume of the collection is concerned only with one type, hunters,
which require:

\begin{itemize}
  \item at least one hunt bell,
  \item all the working bells to do the same work in the plain course,
  \item the number of leads to be the same as the number of working bells.
\end{itemize}

In addition it is restricted to non-little Plain methods where either

\begin{enumerate}
  \item there is a single plain hunt bell and the method is symmetrical about
  its lead and lie (palindromic symmetry)

  or

  \item there are two plain hunt bells and the method is capable of producing an
  extent.
\end{enumerate}

In each case the blue line of the method, details of standard calls,
and a selection of compositions are given.
Compositions of spliced are also included for methods in group a) above.

\section{Place Notation} \label{sec:place-notation}

Each method can be described via its place notation.
In Doubles either one or three places have to be made between each change and
the next, and the place notation system used on the following pages shows just
which places should be made at each change.

The notation of Plain Bob (No. 1) is:

\begin{verbatim}
5 1 5 1 5 1 5 1 5 125
\end{verbatim}

This means that, starting from rounds,
5ths place is made to produce the second row, 21435;
then 1sts to give the next, 24153;
then 5ths for 42513;
and so on until the last of five 5ths has been made,
at which point 13254 will have been obtained.
Finally places made in 1sts, 2nds and 5ths ({\tt 125}) give 13524.
The whole process is then repeated to give the second lead, ending with 15432;
and so on until 12345 is reached at the end of the plain course.

\section{Methods with a single plain hunt bell} \label{sec:intro-single-hunt}

These methods have one bell which plain hunts throughout the lead.
In all the methods of this type so far named,
this hunt bell has been the treble.
We assume that this will always be the case, noting that any bell can be made
the hunt bell by starting the lead at an appropriate point.
It is helpful to analyse these methods in terms of the places made above and
below the treble,
i.e., to classify the methods by their work above the treble (overwork) and
their work below the treble (underwork).

For doubles methods, there are six possible types of change:
double changes ({\tt 1 3 5}) and single changes ({\tt 123 125 145 345}).
At a particular point in the lead not all these changes will preserve the
treble's plain hunting path.
The valid changes at each row of the method are shown in the diagram below:

\begin{figure}[h]
  \centering
  \includegraphics{figs/single-hunt-grids}
\end{figure}

Considering only changes over the treble and restricting our analysis to
symmetrical (palindromic) overworks,
we find that there are \(3 \times 2 \times 4 = 24\) possible overworks.
(3 changes are possible when the treble is in 1-2;
2 when it is in 2-3; and 4 over the lead-end).
Similarly there are 24 possible underworks.

Some of these over-/underworks cause a bell to lie in the same place for more
than four blows.
This is disallowed by decision (E)A.1.(f) and so we are reduced to only 20
possible combinations.
Combining these 20 overworks with the 20 underworks yields a total of
\(20 \times 20 = 400\) possible palindromic grids where the treble plain-hunts.

Ringing a lead of any of these grids brings us to the lead head
(the row where the treble leads at backstroke).
If we start from rounds, \(I = [12345]\),
and consider the lead head as a permutation, \(P\),
we can see that the first lead head is equivalent to \(P^1 = P.I\),
i.e. the result of applying the permutation \(P\) to rounds.
If we ring another lead of the method then we will arrive at the second lead
head.
This can be calculated by permuting by \(P\) again as follows:

\begin{equation}
P^2 = P.P^1 = P.P.I
\end{equation}

We can calculate subsequent lead heads in a similar manner.
Eventually we will return to rounds.
In the case of Plain Bob this occurs after four leads.
We find that:

\begin{equation}
P^4 = P.P.P.P.I = I
\end{equation}

If \(P^n = I\) we say that \(n\) is the order of the permutation \(P\).
The order of a five-bell lead head can be 1, 2, 3 or 4,
and this corresponds to a plain course of 1, 2, 3 or 4 leads.
A four-lead course has four working bells and one hunt bell (the treble).
A three-lead course has three working bells, the treble plain-hunting,
and an additional hunt bell.
This hunt bell does the same work each lead but this will not be plain hunting.
(E)A.1.(d) states that there must be more hunt bells than working bells so grids
producing one- or two-lead courses are not considered valid methods.

Of the 400 possible grids we find that 47 produce a one-lead course
(rounds is the first lead head),
and a further 133 produce a two-lead course.
This leaves 220 valid methods,
with 122 producing a three-lead course and 98 having a four-lead course.

\begin{center}
  \begin{tabular}{c | r}
    Leads & Grids \\
    \hline
    4 & 98 \\
    3 & 122 \\
    2 & 133 \\
    1 & 47 \\
    \hline \hline
      & 400
  \end{tabular}
\end{center}

The 1980 collection contained 177 of the valid 220 methods,
listed by place notation, and grouped according to their work above the treble.
The same numbering and groupings are used here, but we have included the figures
for a plain course of each method.
Each overwork is afforded two pages.
The methods are generally listed consecutively but it has sometimes been found
necessary to reorder methods slightly due to space constraints.

The missing 43 methods from the 1980 collection are those with 4ths place made
at the lead end, which can be viewed as bobbed leads of other methods.
These methods are numbered from 201 to 243.
Assigning numbers to these methods is complicated by both the 1980 collection
(which uses numbers 178-187 for twin-hunt methods)
and Melvyn J Hiller’s Doubles Methods and Variations
(which uses numbers from 188-197 for some additional twin-hunt methods).

For the methods in this group a \emph{standard call} is one which alters only
the places made while the treble leads.
There are four possibilities:
no places other than that of the treble (commonly called an \emph{omit}),
places made in {\tt 123} (a \emph{single}),
places made in {\tt 125} (an \emph{extreme}),
or places made in {\tt 145} (a \emph{bob}).
For each overwork,
one of these calls will be the same as the plain lead of the method.
Each of the other three calls may be used to produce touches.
The plain lead and each type of call are shown at the end of each group of
methods.

Along with each method number a letter is given indicating the rows produced at
the lead end and lead head of the first lead of the plain course.
A complete list of these codes is given on page nnn,
along with a selection of compositions suitable for each of the lead head
groups.
Page nnn features a “four-way table” which places methods into a grid based on
their over- and underworks.
Methods with the same overwork appear in the same column,
and those with the same underwork in the same row.

\section{Methods with a two plain hunt bells} \label{sec:intro-double-hunt}

All of the rung methods with two plain hunt bells include the treble as one of
these hunt bells,
but then vary as to which other bell rings the plain hunting path.
We again consider the grid for a lead of the method,
firstly where both the treble and the two plain hunt:

\begin{figure}[h]
  \centering
  \includegraphics{figs/twin-hunt-2-grids}
\end{figure}

We can see that there are now
\(3\times2\times1\times1\times2\times3\times2\times1\times1\times 2 = 144\)
possible grids, of which 36 exhibit palindromic symmetry about the intersection
of the hunt bells.
We must discard methods with fewer than three leads in the plain course and
those where a bell makes more than four blows in any one place.
Doing this we find that 38 methods remain,
of which 8 exhibit palindromic symmetry.

The grid shown above demonstrates the changes permitted where the treble and two
ring a plain hunting path.
By starting any of these methods two changes later,
the treble and three will be found to plain hunt.
Decision (E)A.1.(b) states that these two versions of the method are the same
and cannot be named separately
(the start is more a feature of the composition than the method),
but a new method of either variety could be named if its complement had not
already been rung.
The introduction of this rule made constructions such as ``New Grandsire''
(5.1.5.1.5.1.5.1.3.1, the complement of Grandsire) obsolete.

We now consider the situation where the treble and five plain hunt:

\begin{figure}[h]
  \centering
  \includegraphics{figs/twin-hunt-5-grids}
\end{figure}

Even fewer changes are permitted, with only
\(2\times1\times2\times1\times2\times2\times1\times2\times1\times2 = 64\)
possible grids (16 have palindromic symmetry).
Again we eliminate those methods with fewer than three leads,
although this time the hunt bells’ movement makes it impossible for working
bells to make more than four blows in one place.
We are left with 21 methods, of which only 3 exhibit palindromic symmetry.
Again we can start the lead at a different point, and this technique can be used
to arrange for the treble and four to be plain hunting.

The place notation of each of these pairs of complementary methods is listed,
with palindromic methods listed first.
Methods already rung named at the time of publication are displayed more fully,
with the figures for an entire course provided.
