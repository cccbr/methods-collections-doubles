\chapter{Introduction} \label{cha:introduction}

The purpose of this Collection is to give a definitive listing of Plain Doubles
methods.
It replaces the 1980 Plain Doubles Methods and Variations.

The Central Council's Decision (E) defines four types of method.
This volume of the collection is concerned only with one type, hunters,
which require:

\begin{itemize}
  \item at least one hunt bell,
  \item all the working bells to do the same work in the plain course,
  \item the number of leads to be the same as the number of working bells.
\end{itemize}

In addition it is restricted to non-little Plain methods where either

\begin{enumerate}
  \item there is a single plain hunt bell and the method is symmetrical about
  its lead and lie (palindromic symmetry)

  or

  \item there are two plain hunt bells and the method is capable of producing an
  extent.
\end{enumerate}

In each case the blue line of the method, details of standard calls,
and a selection of compositions are given.
Compositions of spliced are also included for methods in group a) above.

\section{Place Notation} \label{sec:place-notation}

Each method can be described via its place notation.
In Doubles either one or three places have to be made between each change and
the next, and the place notation system used on the following pages shows just
which places should be made at each change.

The notation of Plain Bob (No. 1) is:

\begin{verbatim}
5 1 5 1 5 1 5 1 5 125
\end{verbatim}

This means that, starting from rounds,
5ths place is made to produce the second row, 21435;
then 1sts to give the next, 24153;
then 5ths for 42513;
and so on until the last of five 5ths has been made,
at which point 13254 will have been obtained.
Finally places made in 1sts, 2nds and 5ths ({\tt 125}) give 13524.
The whole process is then repeated to give the second lead, ending with 15432;
and so on until 12345 is reached at the end of the plain course.

\section{Methods with a single plain hunt bell} \label{sec:intro-single-hunt}

These methods have one bell which plain hunts throughout the lead.
In all the methods of this type so far named,
this hunt bell has been the treble.
We assume that this will always be the case, noting that any bell can be made
the hunt bell by starting the lead at an appropriate point.
It is helpful to analyse these methods in terms of the places made above and
below the treble,
i.e., to classify the methods by their work above the treble (overwork) and
their work below the treble (underwork).

For doubles methods, there are six possible types of change:
double changes ({\tt 1 3 5}) and single changes ({\tt 123 125 145 345}).
At a particular point in the lead not all these changes will preserve the
treble's plain hunting path.
The valid changes at each row of the method are shown in the diagram below:


